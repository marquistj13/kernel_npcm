% Created 2016-11-25 周五 08:42
\documentclass[journal,onecolumn]{IEEEtran}
\usepackage[utf8]{inputenc}
\usepackage[T1]{fontenc}
\usepackage{fixltx2e}
\usepackage{graphicx}
\usepackage{longtable}
\usepackage{float}
\usepackage{wrapfig}
\usepackage{rotating}
\usepackage[normalem]{ulem}
\usepackage{amsmath}
\usepackage{textcomp}
\usepackage{marvosym}
\usepackage{wasysym}
\usepackage{amssymb}
\usepackage{hyperref}
\tolerance=1000
\usepackage[thmmarks, amsmath, thref]{ntheorem}
\theoremstyle{definition}
\makeatletter \renewtheoremstyle{plain} {\item{\theorem@headerfont ##1\ ##2\theorem@separator}~}  {\item{\theorem@headerfont ##1\ ##2\ (##3)\theorem@separator}~}
\theoremheaderfont{\normalfont\bfseries}
\theoremseparator{:}
\theorembodyfont{\normalfont}
\theoremsymbol{\ensuremath{\blacksquare}}
\newtheorem{definition}{Definition}
\newtheorem*{proof}{Proof}
\newtheorem{prop}{Proposition}
\usepackage[caption=false,font=footnotesize]{subfig}
\usepackage{algorithm}
\usepackage{algpseudocode}
\renewcommand{\algorithmicrequire}{\textbf{Input:}}
\newcommand{\crhd}{\raisebox{.25ex}{$\rhd$}}
\renewcommand{\algorithmiccomment}[1]{{\hspace{-0.6cm}$\crhd$ {\it {#1}}}}
\interdisplaylinepenalty=2500
\date{}
\title{Kernel NPCM}
\hypersetup{
  pdfkeywords={},
  pdfsubject={},
  pdfcreator={Emacs 24.5.1 (Org mode 8.2.10)}}
\begin{document}

\maketitle
\begin{abstract}
KNPCM
\end{abstract}
\section{Algorithm}
\label{sec-1}
\subsection{Initialization}
\label{sec-1-1}
Let $A_j^{ini}$ be the set of points $\mathbf{x}_i$ that belong to cluster $C_j$ and $n_j$ be the size of $A_j^{ini}$. Then the we set
\begin{IEEEeqnarray}{ll}
\phi(\boldsymbol{\theta}_j) &= \frac{\Sigma_{i}\phi(\mathbf{x}_i)}{n_j}  \quad \text{for}\;\mathbf{x}_i \in A_j^{ini} \label{npcm_ini_theta}\\
u_{ij} &= 1 \quad \text{if}\;\mathbf{x}_i\in A_j^{ini}
\label{theta_ini}
\end{IEEEeqnarray}
Then initialize $\eta_j$
\begin{IEEEeqnarray}{ll}
\eta_j &= \frac{1}{n_j}\sum_{\mathbf{x}_i \in A_j^{ini}}\|\phi(x_i)-\phi(\theta_j)\| \\
       &= \frac{\sum_{\mathbf{x}_i}\sqrt{\phi(x_i)\cdot\phi(x_i)-2\phi(x_i)\cdot\phi(\theta_j)+\phi(\theta_j)\cdot\phi(\theta_j)}}{n_j} \\
&=\frac{1}{n_j} \sum_i\sqrt{\phi(x_i)\cdot\phi(x_i)-2\phi(x_i)\cdot\frac{\Sigma_{m}\phi(x_m)}{n_j}+\frac{\Sigma_{mn}\phi(x_m)\cdot\phi(x_n)}{n_j^2}}\\
&=\frac{1}{n_j} \sum_i\sqrt{k(x_i,x_i)-2\frac{\Sigma_{m}k(x_i,x_m)}{n_j}+\frac{\Sigma_{mn}k(x_m,x_n)}{n_j^2}}
\end{IEEEeqnarray}
\subsection{Update}
\label{sec-1-2}
The update of $\theta_j$
\begin{equation}
\phi(\boldsymbol{\theta}_j)=\frac{\Sigma_{i=1}^Nu_{ij}\phi(\mathbf{x}_i)}{\Sigma_{i=1}^Nu_{ij}} \quad \text{for}\;u_{ij}\geq \alpha.
\label{theta_update}
\end{equation}
For the update of $U$, we first calculate $d_{ij}$
\begin{IEEEeqnarray}{ll}
d_{ij}$ &= \|\phi(x_i)-\phi(\theta_j)\| \\
&=\sqrt{\phi(x_i)\cdot\phi(x_i)-2\phi(x_i)\cdot\frac{\Sigma_{m}u_{mj}\phi(x_m)}{\Sigma_{m}u_{mj}}+\frac{\Sigma_{mn}u_{mj}u_{nj}\phi(x_m)\cdot\phi(x_n)}{(\Sigma_{m}u_{mj})^2}}\\
       &= \sqrt{k(x_i,x_i)-2\frac{\Sigma_{m}u_{mj}k(x_i,x_m)}{\Sigma_{m}u_{mj}}+\frac{\Sigma_{mn}u_{mj}u_{nj}k(x_m,x_n)}{(\Sigma_{m}u_{mj})^2}}
\end{IEEEeqnarray}
Then set 
\begin{equation}
\label{npcm_u_update}
\mu_{ij}=\exp\left(-\frac{d_{ij}^2}{\gamma_j}\right)
\end{equation}
where $\gamma_j=\left(0.5\eta_{j}+0.5\sqrt{\eta_{j}^{2}+0.8d_{ij}\eta_j/\sqrt{-\ln\alpha}}\right)^2$

After updating $U$, $\theta_j$ is updated according to \eqref{theta_update}. The update of $\eta_j$ is 
\begin{equation}
\label{npcm_eta_update}
\eta_j=\frac{1}{n_j}\sum_{\mathbf{x}_i\in A_j}||\phi(x_i)-\phi(\theta_j)|| \quad \text{for}\;u_{ij} \geq 0.01.
\end{equation}
where $A_j$ and $n_j$ have the same meaning as in \eqref{theta_ini}.
% Emacs 24.5.1 (Org mode 8.2.10)
\end{document}
